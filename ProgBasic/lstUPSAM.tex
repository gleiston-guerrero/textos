%========================================================
% Listados UPSAM 2.0 con colores (lstlisting)
%========================================================
% Requiere en el preámbulo:
% \usepackage{xcolor}
% \usepackage{listings}
%
% (Opcional pero recomendable para acentos en comentarios/strings)
% \usepackage[T1]{fontenc}
% \usepackage[utf8]{inputenc} % si NO compila con LuaLaTeX/XeLaTeX
%========================================================

%-----------------------------
% Paleta de colores (ajustable)
%-----------------------------
\definecolor{upsamKeyword}{RGB}{0,76,153}     % palabras reservadas
\definecolor{upsamType}{RGB}{128,0,128}       % tipos (entero, real, cadena, lógico)
\definecolor{upsamId}{RGB}{0,102,0}           % identificadores
\definecolor{upsamInt}{RGB}{153,76,0}         % constantes enteras
\definecolor{upsamReal}{RGB}{153,0,76}        % constantes reales
\definecolor{upsamString}{RGB}{163,21,21}     % cadenas
\definecolor{upsamComment}{RGB}{120,120,120}  % comentarios { ... }
\definecolor{upsamOp}{RGB}{0,0,0}             % operadores
\definecolor{upsamBg}{RGB}{250,250,250}       % fondo suave
\definecolor{upsamFrame}{RGB}{220,220,220}    % marco

%--------------------------------------------------------
% Definición del lenguaje UPSAM (lstdefinelanguage)
%--------------------------------------------------------
\lstdefinelanguage{UPSAM}{
	sensitive=true,
	
	% Comentarios entre llaves: { ... }
	morecomment=[s]{\{}{\}},
	commentstyle=\color{upsamComment}\itshape,
	
	% Cadenas entre comillas
	morestring=[b]",
	stringstyle=\color{upsamString},
	
	% Palabras reservadas (control, E/S, estructura)
	morekeywords=[1]{
		ALGORITMO,PROGRAMA,VAR,INICIO,FIN,
		SI,ENTONCES,SINO,FINSI,FIN_SI,
		MIENTRAS,HAGA,FINMIENTRAS,FIN_MIENTRAS,
		PARA,HASTA,CON,PASO,FINPARA,FIN_PARA,
		REPETIR,HASTAQUE,HASTA_QUE,
		LEER,ESCRIBIR,
		FUNCION,FUNCTION,RETORNAR,DEVOLVER,FINFUNCION,FIN_FUNCION
	},
	keywordstyle=[1]\color{upsamKeyword}\bfseries,
	
	% Tipos y literales lógicos
	morekeywords=[2]{ENTERO,REAL,CADENA,LOGICO,LÓGICO,VERDADERO,FALSO},
	keywordstyle=[2]\color{upsamType}\bfseries,
	
	% Operadores/relacionales frecuentes (para destacarlos)
	morekeywords=[3]{MOD,DIV,Y,O,NO},
	keywordstyle=[3]\color{upsamOp}\bfseries,
	
	% Reglas para números (enteros y reales)
	% (listings no separa "real" vs "entero" por defecto; se consigue con "literate")
	alsoletter={.},
	literate=
	% Reales (con punto decimal)
	{0.0}{{\textcolor{upsamReal}{0.0}}}3
	{0.1}{{\textcolor{upsamReal}{0.1}}}3
	{0.2}{{\textcolor{upsamReal}{0.2}}}3
	{0.3}{{\textcolor{upsamReal}{0.3}}}3
	{0.4}{{\textcolor{upsamReal}{0.4}}}3
	{0.5}{{\textcolor{upsamReal}{0.5}}}3
	{0.6}{{\textcolor{upsamReal}{0.6}}}3
	{0.7}{{\textcolor{upsamReal}{0.7}}}3
	{0.8}{{\textcolor{upsamReal}{0.8}}}3
	{0.9}{{\textcolor{upsamReal}{0.9}}}3
	% Operador de asignación UPSAM 2.0
	{<-}{{\textcolor{upsamOp}{<-}}}2
	% Operadores lógicos estilo C/C++ presentes en ejemplos comparativos
	{||}{{\textcolor{upsamOp}{||}}}2
	{|}{{\textcolor{upsamOp}{|}}}1
	{&}{{\textcolor{upsamOp}{\&}}}1
	{^}{{\textcolor{upsamOp}{\^{}}}}1
	{<<}{{\textcolor{upsamOp}{<<}}}2
	{>>}{{\textcolor{upsamOp}{>>}}}2
}

%--------------------------------------------------------
% Estilo global para listados UPSAM
% - Identificadores (variables, nombres) en otro color
%--------------------------------------------------------
\lstdefinestyle{upsamStyle}{
	language=UPSAM,
	basicstyle=\ttfamily\small,
	backgroundcolor=\color{upsamBg},
	frame=single,
	rulecolor=\color{upsamFrame},
	frameround=ffff,
	framesep=6pt,
	
	% Identificadores: se colorean con identifierstyle
	identifierstyle=\color{upsamId},
	
	% Números: se colorean por "numbersstyle" solo si se activan números de línea;
	% para constantes enteras se usa el estilo "upsamInt" vía regla de "literate" parcial (ver abajo).
	% Para enteros, se aplica un truco: resaltar dígitos individuales.
	literate=
	*{0}{{\textcolor{upsamInt}{0}}}1
	{1}{{\textcolor{upsamInt}{1}}}1
	{2}{{\textcolor{upsamInt}{2}}}1
	{3}{{\textcolor{upsamInt}{3}}}1
	{4}{{\textcolor{upsamInt}{4}}}1
	{5}{{\textcolor{upsamInt}{5}}}1
	{6}{{\textcolor{upsamInt}{6}}}1
	{7}{{\textcolor{upsamInt}{7}}}1
	{8}{{\textcolor{upsamInt}{8}}}1
	{9}{{\textcolor{upsamInt}{9}}}1
	% Mantener las reglas del lenguaje (reales y operadores) por encima:
	{0.0}{{\textcolor{upsamReal}{0.0}}}3
	{0.1}{{\textcolor{upsamReal}{0.1}}}3
	{0.2}{{\textcolor{upsamReal}{0.2}}}3
	{0.3}{{\textcolor{upsamReal}{0.3}}}3
	{0.4}{{\textcolor{upsamReal}{0.4}}}3
	{0.5}{{\textcolor{upsamReal}{0.5}}}3
	{0.6}{{\textcolor{upsamReal}{0.6}}}3
	{0.7}{{\textcolor{upsamReal}{0.7}}}3
	{0.8}{{\textcolor{upsamReal}{0.8}}}3
	{0.9}{{\textcolor{upsamReal}{0.9}}}3
	{<-}{{\textcolor{upsamOp}{<-}}}2
	{||}{{\textcolor{upsamOp}{||}}}2
	{|}{{\textcolor{upsamOp}{|}}}1
	{&}{{\textcolor{upsamOp}{\&}}}1
	{^}{{\textcolor{upsamOp}{\^{}}}}1
	{<<}{{\textcolor{upsamOp}{<<}}}2
	{>>}{{\textcolor{upsamOp}{>>}}}2,
	
	showstringspaces=false,
	tabsize=3,
	keepspaces=true,
	columns=fullflexible,
	
	% Opcional: números de línea
	numbers=left,
	numberstyle=\scriptsize\color{upsamComment},
	numbersep=10pt,
	
	% Saltos de línea en listados largos
	breaklines=true,
	breakatwhitespace=true,
	
	% Títulos (caption) consistentes
	captionpos=b
}
